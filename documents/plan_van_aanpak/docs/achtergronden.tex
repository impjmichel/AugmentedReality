% A chapter named 'Your first document' is created

\chapter{Achtergronden} \label{cha:your-first-document}

\section{Opdracht} \label{sec:opdracht}

Dit project is een \underline{augmented reality} project in opdracht van Avans Hogeschool Breda, als onderdeel van de opleiding \textbf{Technisch Informatica}. Het project maakt gebruik van OpenCV voor motion detection en OpenGL voor de 3D graphics. Dit project zal computer gegenereerde beelden toevoegen aan reele beelden van de fysieke werkelijkheid. Het project zal worden geprogrammeerd in C en C++.

Text is formatted with: \textbf{bold}, \textit{italic} and \underline{underline}.
\Cref{sec:basics} is part of \cref{cha:your-first-document}.

\section{Projectnaam} \label{sec:projectnaam}
Het project is genaamd: "\projectname"

\section{Opdrachtgever} \label{sec:opdrachtgever}
Dit schoolproject heeft geen officiele opdrachtgever. De actuele opdrachtgever is dus Avans Hogeschool Breda. De eindbeoordeling zal worden gedaan door de periodecoordinator en andere docenten.

\section{Organisatie} \label{sec:organisatie}
Het ontwikkelteam bestaat uit 5 Technisch Informatica studenten.
\begin{table}[h]
  \centering
  \caption{Organisatie}
  \label{tb:table}
  \begin{tabular}{crl}
    \toprule
    Naam     		& Rol			& Studentnr\\
    \midrule
    Johannes Michel	& Gitmaster		& 2060486\\
    Robbert van Nijnatten	& Projectleider	& 2052820\\
    Raymond Rohder	& Projectleider	& 1115099\\
    Vincent Stout	& Secretaris	& 2066962\\
    Kevin van der Vleuten	& Bugtracking	& 2059022\\
    \bottomrule
  \end{tabular}
\end{table}

\section{Stakeholders} \label{sec:stakeholders}
De balanghebbenden van dit project zijn uiteraard de \textbf{gebruikers}. Zij leren de wereld en haar monumenten kennen door de applicatie te gebruiken. Daarnaast leren de \textbf{studenten} van het ontwikkelteam werken met 3D graphics, motion detection en augmented reality. \textbf{Basisscholen} kunnen het eindproduct ook toepassen tijdens topografie lessen.

\section{Document structuur} \label{sec:structure}
Dit document is opgebouwd uit de volgende hoofdstukken.
\begin{enumerate}
	\item Achtergronden
	\item Projectresultaat
	\item Projectactiviteiten
	\item Projectgrenzen
	\item Tussenresulten
	\item Kwaliteit
	\item Projectorganisatie
	\item Planning
	\item Kosten en baten
	\item Risico's
\end{enumerate}