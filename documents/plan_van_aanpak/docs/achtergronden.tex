% A chapter named 'Your first document' is created

\chapter{Achtergronden} \label{cha:achtergrond}

\section{Opdracht} \label{sec:opdracht}

Dit document gaat over een augmented reality project in opdracht van Avans Hogeschool Breda, als onderdeel van de opleiding Technisch Informatica. Er moet gebruik worden gemaakt van OpenCV voor motion detection en OpenGL voor de 3D graphics. Dit project zal computer gegenereerde beelden toevoegen aan reele beelden van de fysieke werkelijkheid. Het programmeren zal voornamelijk in C en C++ worden gedaan.

\section{Projectnaam} \label{sec:projectnaam}
Het project is genaamd: \projectname

\section{Opdrachtgever} \label{sec:opdrachtgever}
GloMon heeft geen officiele opdrachtgever. De actuele opdrachtgever is dus Avans Hogeschool Breda. De eindbeoordeling zal worden gedaan door de periodecoordinator en andere docenten.

\section{Organisatie} \label{sec:organisatie}
De organisatie bestaat uit een klant (hier de periode eigenaar), een innovatie en ontwikkelteam die het idee bedenken en maken en een senior die het ontwikkelteam ondersteunt.

\section{Stakeholders} \label{sec:stakeholders}
De belanghebbenden van dit project zijn uiteraard de \textbf{gebruikers}. Zij leren de wereld en haar monumenten kennen door de applicatie te gebruiken. Daarnaast leren de \textbf{studenten} van het ontwikkelteam werken met 3D graphics, motion detection en augmented reality. \textbf{Basisscholen} kunnen het eindproduct ook toepassen tijdens topografie lessen.

\section{Document structuur} \label{sec:structure}
Dit document bevat meerdere hoofdstukken, om de lezer een idee te geven van wat er gaat komen zal hier een korte toelichting per hoofdstuk gegeven worden. In achtergronden zal het doel van de opdracht worden besproken. Het projectresultaat zal een inzicht geven over de uiteindelijke applicatie. Een opsomming van alle sprints zal worden weergegeven in de projectactiviteiten. De projectgrenzen geven aan wat wel en wat niet te verwachten van de applicatie. In het hoofdstuk tussenresultaten zal een lijst staan van alle producten die tussentijds moeten worden opgeleverd. In kwaliteit wordt verteld wat de kwaliteitseisen zijn van de applicatie. Onder het kopje projectorganisatie zal worden uitgelegd wie welke rol heeft in de projectgroep. De planning geeft alle deadlines weer. Om te zien wat het project zal gaan kosten en wat het oplevert is er een kopje kosten en baten. Ten slotte is er nog een hoofdstuk Risico's waar alle interne en externe risico's worden opgesomd.