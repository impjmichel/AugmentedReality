\chapter{Kwaliteit} 
Elk project kan in de loop van een traject de oorspronkelijke doelen, waarden en functionaliteiten uit het oog verliezen. Daarom is de kwaliteitsbewaking noodzakelijk voor elk project. Hoe dit gebeurt binnen het traject van \projectname\ zal in dit hoofdstuk uitgelegd worden.

De kwaliteit van elk onderdeel, en hierdoor de kwaliteit van het eindproduct, wordt bewaakt door verschillende methodes. Door het juiste gebruik van elk van deze handelwijzen zal het product uitgroeien tot een volwaardig e-learning tool.

Het groepsproces wordt bewaakt doormiddel van Scrum. De meetings vinden plaats in de ochtend van elke projectdag en hebben als doel de voortgang vast te leggen en te bewaken.

De methode XtremeProgramming bewaakt het programmeerproces doordat er tijdens het programmeren lopend een review plaatsvindt. Hierdoor kan foutieve code meteen ontdekt worden voordat er een grootschalige test uitgevoerd hoeft te worden.
Verder wordt de code bewaakt door het handhaven van een afgesproken codestandaard.

De versiebewaking gebeurt doormiddel van Git en het gebruik van verschillende codetakken. Deze takken zullen het programmeerwerk overzichtelijker maken en tegelijkertijd veiligheid bieden doordat alleen na een review code naar een andere tak overgezet kan worden.

Voor het afhandelen van fouten in de applicatie wordt het Mantis ticketsysteem gebruikt. Dit voorkomt dat meerdere personen aan dezelfde oplossing werken en geeft verder een goede overzicht van alle taken die nog uitgevoerd moeten worden.

Al deze methodes zorgen voor een goed verloop van de activiteiten en zijn tevens opgenomen in het samenwerkingscontract. Dit zal latere conflicten vroegtijdig afkappen en voor een beter proces leiden.
