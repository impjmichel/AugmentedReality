\chapter{Risico's} \label{cha:risicos}

Zoals bij elk project heeft dit project ook risico’s, deze risico’s zouden het project kunnen beïnvloeden waardoor het gedeeltelijk of geheel kan falen. De risico’s die zich bij dit project mogelijk kunnen voordoen zijn:

\section{Interne risico's} \label{sec:Interne risicos}

% Inserting a table

    \begin{tabular}{ | l | l | l | }
    	\hline
    	Nr. & Algemeen & Schaal(1 t/m 10) \\ \hline
    	1.	& Ziekteverzuim & 8 \\ \hline
    	2.  & Afwijken van de planning (tijdsnood) & 5 \\ \hline
    	3.  & Conflicten binnen de projectgroep & 7 \\ \hline
    	4.  & Miscommunicatie binnen de projectgroep & 6 \\ \hline
    	5.  & Product niet volgens de eisen van de opdrachtgever & 5 \\ \hline
  	\end{tabular}
\\
\\
    \begin{tabular}{ | l | l | l | }
    	\hline
    	Nr. & Kennis & Schaal(1 t/m 10) \\ \hline
    	1.	& Te weinig kennis over het project & 4 \\ \hline
    	2.  & Te weinig ervaring & 5 \\ \hline
  	\end{tabular}
\\
\\
    \begin{tabular}{ | l | l | l | }
    	\hline
    	Nr. & Technisch & Schaal(1 t/m 10) \\ \hline
    	1.	& Instabiele werkomgeving & 3 \\ \hline
    	2.  & Data verlies & 5 \\ \hline
    	3.  & Data niet bereikbaar & 7 \\ \hline
  	\end{tabular}

\section{Externe risico's} \label{sec:Externe risicos}

    \begin{tabular}{ | l | l | l | }
    	\hline
    	Nr. & Algemeen & Schaal(1 t/m 10) \\ \hline
    	1.	& Afwezigheid van de opdrachtgever & 4 \\ \hline
    	2.  & Miscommunicatie tussen de opdrachtgever en de projectgroep & 6 \\ \hline
  	\end{tabular}
\\
\\
Opmerking: Om zo min mogelijk door risico’s het project te laten beïnvloeden is er in de planning extra ruimte gecreëerd om eventuele achterstanden in te halen.