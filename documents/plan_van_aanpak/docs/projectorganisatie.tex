% A chapter named 'Your first document' is created
\chapter{Projectorganisatie} \label{cha:projectorganisatie}


% A section called 'Basics' is created
\section{Rollen} \label{sec:basics}

\textbf{Projectleiders en projectplanners(Council)}\\
\textbf{naam:}	Raymond Rohder\\
\textbf{email:}	rchrohde@student.avans.nl

\bigskip
\textbf{naam:}	Robbert van Nijnatten\\
\textbf{email:}	rvnijnatten@casema.nl

\bigskip
De taak van de projectleider is het bewaren van overzicht in het team. Hij is het aanspreekpunt voor alles aangaande het product, en verdeelt de taken binnen het team. De projectleider maakt gebruik van verslagen van de andere projectleden om zijn of haar werk goed te doen. De projectleider rapporteert rechtstreeks aan de senior met de producten en bevindingen van het team.

\bigskip
\textbf{Projectsecretaris}\\
\textbf{naam:}	Vincent Stout\\
\textbf{email:}	vastout@gmail.com

\bigskip
De secretaris is de rechterhand van de projectleider. Hij zorgt ervoor dat het document op tijd wordt verwerkt en stuurt deze na goedkeuring van de projectleider op. De secretaris zorgt er ook voor dat de logboeken individueel worden bijgehouden en waarschuwt als er documenten ontbreken. De secretaris is ook verantwoordelijk voor de indeling van de mappenstructuur.
De Secretaris schrijft ook de notulen tijdens de vergaderingen.

\bigskip
\textbf{Projectversiebeheerder}\\
\textbf{naam:}	Johannes Michel\\
\textbf{email:}	johannes.san@gmail.com

\bigskip
De versiebeheerder zorgt ervoor dat alle code en document opgeslagen worden met behulp van versiebeheer website zoals github.com. Hier maakt hij een repository aan die weer verdeeld wordt in branches. Elke week merged hij alle branches naar de masterbranch.

\bigskip
\textbf{Projecttester}\\
\textbf{naam:}	Kevin van der Vleuten\\
\textbf{email:}	kevin.vd.vleuten@gmail.com

\bigskip
De projecttester test de gemaakte applicatie een keer per week. Tevens houd hij alle bugs bij in een bugtracker zoals Mantis. Pas na goedkeuring van de projecttester mag de versiebeheerder de code mergen.

\bigskip
\textbf{Projectsenior}\\
\textbf{naam:}	Diederich Kroeske\\
\textbf{email:}

\bigskip
De projectsenior houd de voortgang van het project nauwlettend in de gaten. De senior is ook altijd bereikbaar voor vragen.

% A subsection named 'Typesetting content' is created
\section{Beschikbaarheid} \label{sec:typesetting}
Op projectdagen dienen alle leden beschikbaar te zijn. De projectdagen zijn:
\begin{table}[h]
  \label{tb:table}
  \begin{tabular}{crl}
    \toprule
    Dag     & 			datum & 		tijden    \\
    \midrule
    vrijdag     & 25 	april 	2014   & 9:30-17:00\\
    vrijdag     & 2 	mei 	2014   & 9:30-17:00\\
    vrijdag     & 9 	mei 	2014   & 9:30-17:00\\
    vrijdag     & 16 	mei 	2014   & 9:30-17:00\\
    vrijdag     & 23 	mei 	2014   & 9:30-17:00\\
    vrijdag     & 30 	mei 	2014   & 9:30-17:00\\
    vrijdag     & 6 	juni 	2014   & 9:30-17:00\\
    \bottomrule
  \end{tabular}
\end{table}