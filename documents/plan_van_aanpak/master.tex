% This is the master file of the folder structure. In order to compile your document, run this file. In most LaTeX editors, the master file can be specified such that the document can also be compiled from the other .tex files (in the docs folder).

% First, the preamble needs to be called. This contains all the 'under the hood' stuff for your document.
\input{docs/preamble}

% The title page is created with the command \maketitle which needs to be placed after the \begin{document} command. To create the titlepage, some entries are needed: the name of the autor is defined by \author{}, the title by the entry \title{} and the date by the command \date{}. Note that the current date is displayed with \today.
\author{Johannes Michel\\Robbert van Nijnatten\\Raymond Rohder\\Vincent Stout\\Kevin van der Vleuten}
\title{Plan van aanpak - \projectname}
\date{\today}

% All the actual content of your document should be placed after \begin{document} and before \end{document}. This content should be placed in the docs folder and can then be called with \input{docs/filename}.
\begin{document}

% Here the actual title page is printed, based on the given entries \author{}, \title{} and \date{}.
\maketitle

% The table of contents can be automatically generated with the \tableofcontents command. Note that you need to compile the document twice in order to see the changes in the table of contents.
\renewcommand*\contentsname{Inhoud}
\tableofcontents

% The \input{} command reads and processes the indicated example.tex file. Note that docs/ locates the folder where the .tex file is stored.
%\input{docs/example}
% A chapter named 'Your first document' is created

\chapter{Achtergronden} \label{cha:your-first-document}

\section{Opdracht} \label{sec:opdracht}

Dit project is een \underline{augmented reality} project in opdracht van Avans Hogeschool Breda, als onderdeel van de opleiding \textbf{Technisch Informatica}. Het project maakt gebruik van OpenCV voor motion detection en OpenGL voor de 3D graphics. Dit project zal computer gegenereerde beelden toevoegen aan reele beelden van de fysieke werkelijkheid. Het project zal worden geprogrammeerd in C en C++.

Text is formatted with: \textbf{bold}, \textit{italic} and \underline{underline}.
\Cref{sec:basics} is part of \cref{cha:your-first-document}.

\section{Projectnaam} \label{sec:projectnaam}
Het project is genaamd: "\projectname"

\section{Opdrachtgever} \label{sec:opdrachtgever}
Dit schoolproject heeft geen officiele opdrachtgever. De actuele opdrachtgever is dus Avans Hogeschool Breda. De eindbeoordeling zal worden gedaan door de periodecoordinator en andere docenten.

\section{Organisatie} \label{sec:organisatie}
Het ontwikkelteam bestaat uit 5 Technisch Informatica studenten.
\begin{table}[h]
  \centering
  \caption{Organisatie}
  \label{tb:table}
  \begin{tabular}{crl}
    \toprule
    Naam     		& Rol			& Studentnr\\
    \midrule
    Johannes Michel	& Gitmaster		& 2060486\\
    Robbert van Nijnatten	& Projectleider	& 2052820\\
    Raymond Rohder	& Projectleider	& 1115099\\
    Vincent Stout	& Secretaris	& 2066962\\
    Kevin van der Vleuten	& Bugtracking	& 2059022\\
    \bottomrule
  \end{tabular}
\end{table}

\section{Stakeholders} \label{sec:stakeholders}
De balanghebbenden van dit project zijn uiteraard de \textbf{gebruikers}. Zij leren de wereld en haar monumenten kennen door de applicatie te gebruiken. Daarnaast leren de \textbf{studenten} van het ontwikkelteam werken met 3D graphics, motion detection en augmented reality. \textbf{Basisscholen} kunnen het eindproduct ook toepassen tijdens topografie lessen.

\section{Document structuur} \label{sec:structure}
Dit document is opgebouwd uit de volgende hoofdstukken.
\begin{enumerate}
	\item Achtergronden
	\item Projectresultaat
	\item Projectactiviteiten
	\item Projectgrenzen
	\item Tussenresulten
	\item Kwaliteit
	\item Projectorganisatie
	\item Planning
	\item Kosten en baten
	\item Risico's
\end{enumerate}
\chapter{Projectresultaat} \label{cha:projectresultaat}
Het projectresultaat omschrijft de situatie wanneer PROJECT zijn doel heeft bereikt. Dit aan de hand van een doelstelling en eindvoorwaarden.
\section{Doelstelling} \label{sec:doelstelling}
Het doel van PROJECT is een proof of concept waarbij Augmented Reality centraal staat. De gebruiker komt in aanraking met een met extra informatie aangevulde realiteit.
\section{Eindvoorwaarden} \label{sec:eindvoorwaarden}
\begin{enumerate}
  \item Interactiviteit. Interactie tussen de gebruiker en PROJECT bepaald die inhoud van getoonde informatie.
  \item Computervision. Acties van de gebruiker worden met behulp van een camera vastgelegd en gebruikt als input voor de applicatie.
  \item Informatief. Alle getoonde informatie is dynamisch, contextueel relevant en afgestemd op de gebruiker.
  \item 3D Graphics. De user interface bevat 3D beelden. 
\end{enumerate}
\chapter{Projectactiviteiten} \label{cha:projectactiviteiten}
Ter voorbereiding van dit project zijn hieronder puntsgewijs de te verwachten projectactiviteiten opgesomd.
\section{Vooronderzoek (Sprint 1 - Week 1)} \label{sec:vooronderzoek}
\begin{enumerate}
  \item Brainstormsessie t.b.v. projectopdracht
  \item Resultaat uitwerken tot concept ontwerp
\end{enumerate}
\section{Voorbereiding (Sprint 2 - Week 2)} \label{sec:voorbereiding}
\begin{enumerate}
  \item Opstellen van het Plan van Aanpak en planning
  \item Reviewen en aanpassen van het Plan van Aanpak en planning
  \item Opstellen van het Interaction Design document
  \item Reviewen en aanpassen van het Interaction Design document
\end{enumerate}
\section{Ontwerp (Sprint 3 - Week 3)} \label{sec:ontwerp}
\begin{enumerate}
  \item Opstellen van het Technische ontwerp (klassendiagram)
  \item Reviewen en aanpassen van het Technische ontwerp
\end{enumerate}
\section{Proof of Concept (Sprint 4 - Week 4 en 5)} \label{sec:proofofconcept}
\begin{enumerate}
  \item Uitwerken van het Technische ontwerp tot proof of concept
  \item Onderdelen samenvoegen tot één product
\end{enumerate}
\section{Testen (Sprint 5 - Week 6 en 7)} \label{sec:testen}
\begin{enumerate}
  \item Proof of concept testen op fouten (Debuggen)
  \item Uitwerken van proof of concept tot definitief product
\end{enumerate}
\section{Afronding (Sprint 6 - Week 8)} \label{sec:afronding}
\begin{enumerate}
  \item Opstellen van presentatie en promotie video
  \item Reviewen en aanpassen van de presentatie en promotie video
  \item Definitief product opleveren
\end{enumerate}
\chapter{Projectgrenzen} \label{cha:projectgrenzen}
Om onduidelijkheden te voorkomen dienen de projectvoorwaarden en werkzaamheden afgebakend te worden. Dit pilot-project is een proof of concept en zal niet op de commerciele markt uitgebracht worden.
\\
\begin{enumerate}
  \item \projectname\ wordt 'as is' opgeleverd wanneer de opleveringsdatum volgens de planning bereikt is.
  \item \projectname\ voorziet de realiteit van extra informatie (Augmented Reality).
  \item \projectname\ wordt ontwikkeld voor het Windows-pc platform.
  \item \projectname\ maakt gebruik van OpenGL (Open Graphics Library) om 3D-graphics te realiseren.
  \item \projectname\ maakt gebruik van OpenCV (Open Source Computer Vision) om real-time computer vision (object herkenning) te realiseren.
\end{enumerate}
\chapter{Tussenresultaten}


\section{Basics} \label{sec:basics}
\paragraph{Tijdens de projectwerkzaamheden ontstaan enkele tussenresultaten die samenhangen aan de projectactiviteiten. }

\subsubsection{Descriptive}
Creating a descriptive list:
\begin{description}
  \item[Interaction Design Document] Dit document bevat een visuele vormgeving van het uiteindelijke product en wordt gebruikt voor de beeldvorming.
  \item[Plan van Aanpak] Een verduidelijking van de projectactiviteiten en benodigdheden van dit project.
  \item[Product Iteratie I] Scrum tussenresultaat van het einde van Sprint 1.
  \item[Product Iteratie II] Scrum tussenresultaat van het einde van Sprint 2.
\end{description}


\paragraph{End of story, conclusion or something! work harder damned imbecile!}
\chapter{Kwaliteit} 

\paragraph{Elk project kan in de loop van een traject de oorspronkelijke doelen, waarden en functionaliteiten uit het oog verliezen. Daarom is de kwaliteitbewaking noodzakelijk voor elk project. Hoe dit gebeurt binnen het traject van \projectname\ zal in dit hoofdstuk uitgelegd worden.}
\paragraph{middenstuk, hoe kwaliteitbewaking}
\paragraph{einde, conclusie}

% A chapter named 'Your first document' is created
\chapter{Projectorganisatie} \label{cha:projectorganisatie}


% A section called 'Basics' is created
\section{Rollen} \label{sec:basics}

\textbf{Projectleiders en projectplanners(Council)}\\
\textbf{naam:}	Raymond Rohder\\
\textbf{email:}	rchrohde@student.avans.nl

\bigskip
\textbf{naam:}	Robbert van Nijnatten\\
\textbf{email:}	rvnijnatten@casema.nl

\bigskip
De taak van de projectleider is het bewaren van overzicht in het team. Hij is het aanspreekpunt voor alles aangaande het product, en verdeelt de taken binnen het team. De projectleider maakt gebruik van verslagen van de andere projectleden om zijn of haar werk goed te doen. De projectleider rapporteert rechtstreeks aan de senior met de producten en bevindingen van het team.

\bigskip
\textbf{Projectsecretaris}\\
\textbf{naam:}	Vincent Stout\\
\textbf{email:}	vastout@gmail.com

\bigskip
De secretaris is de rechterhand van de projectleider. Hij zorgt ervoor dat het document op tijd wordt verwerkt en stuurt deze na goedkeuring van de projectleider op. De secretaris zorgt er ook voor dat de logboeken individueel worden bijgehouden en waarschuwt als er documenten ontbreken. De secretaris is ook verantwoordelijk voor de indeling van de mappenstructuur.
De Secretaris schrijft ook de notulen tijdens de vergaderingen.

\bigskip
\textbf{Projectversiebeheerder}\\
\textbf{naam:}	Johannes Michel\\
\textbf{email:}	johannes.san@gmail.com

\bigskip
De versiebeheerder zorgt ervoor dat alle code en document opgeslagen worden met behulp van versiebeheer website zoals github.com. Hier maakt hij een repository aan die weer verdeeld wordt in branches. Elke week merged hij alle branches naar de masterbranch.

\bigskip
\textbf{Projecttester}\\
\textbf{naam:}	Kevin van der Vleuten\\
\textbf{email:}	kevin.vd.vleuten@gmail.com

\bigskip
De projecttester test de gemaakte applicatie een keer per week. Tevens houd hij alle bugs bij in een bugtracker zoals Mantis. Pas na goedkeuring van de projecttester mag de versiebeheerder de code mergen.

\bigskip
\textbf{Projectsenior}\\
\textbf{naam:}	Diederich Kroeske\\
\textbf{email:}

\bigskip
De projectsenior houd de voortgang van het project nauwlettend in de gaten. De senior is ook altijd bereikbaar voor vragen.

% A subsection named 'Typesetting content' is created
\section{Beschikbaarheid} \label{sec:typesetting}
Op projectdagen dienen alle leden beschikbaar te zijn. De projectdagen zijn:
\begin{table}[h]
  \label{tb:table}
  \begin{tabular}{crl}
    \toprule
    Dag     & 			datum & 		tijden    \\
    \midrule
    vrijdag     & 25 	april 	2014   & 9:30-17:00\\
    vrijdag     & 2 	mei 	2014   & 9:30-17:00\\
    vrijdag     & 9 	mei 	2014   & 9:30-17:00\\
    vrijdag     & 16 	mei 	2014   & 9:30-17:00\\
    vrijdag     & 23 	mei 	2014   & 9:30-17:00\\
    vrijdag     & 30 	mei 	2014   & 9:30-17:00\\
    vrijdag     & 6 	juni 	2014   & 9:30-17:00\\
    \bottomrule
  \end{tabular}
\end{table}
% A chapter named 'Your first document' is created
\chapter{Planning} \label{cha:planning}



% Inserting a table
  \begin{tabular}{ | l | l | l |}
    \toprule
    {projectweek}     & {onderdeel} & {deadline}    \\
    \midrule
    week 3  & Document Interaction Design  	& vrijdag 17:00\\
    week 3	& Document Plan van aanpak     	& vrijdag 17:00\\
    week 5	& Oplevering Iteratie product	& vrijdag 17:00\\
    week 6	& Oplevering Iteratie product	& vrijdag 17:00\\
    week 9	& Oplevering produc				& onbekend\\
    \bottomrule
  \end{tabular}


\chapter{Kosten en baten} \label{cha:kostenenbaten}

Dit hoofdstuk beschrijft de kosten en baten van het project GloMon. In de onderstaande tabel zijn alle projectkosten opgenomen die in een periode van 7 weken worden gemaakt.

\section{Kosten} \label{subsec:kosten}

\begin{left}
    \begin{tabular}{ | l | l | l | l | }
    	\hline
    	Nr. &    & Aantal & Kosten \\ \hline
    		& \textbf{Mensuren} & & \\ \hline
    	1.	& Benodigde werktijd per persoon totaal 51 uur & 5 & 6885,00 \\ \hline
    		& \textbf{Hulpmiddelen} & & \\ \hline
    	2.	& Camera & 1 & 450,00 \\ \hline
    	3.	& Wereldbol & 1 & 22,50 \\ \hline
     		& \textbf{Onvoorziene uitgaven} & & \\ \hline
     	4.  & Bijkomende kosten in het project & 1 & 870,00 \\ \hline
     	&	& 	& Totaal: 8227,50 (Incl. BTW) \\ \hline
  	\end{tabular}
\end{left}

\section{Baten} \label{subsec:baten}

Voor dit project zijn er geen baten opgenomen. Dit komt door de minimale hoeveelheid hulpmiddelen dat hierbij gemoeid is. Ook kunnen geen opbrengsten gemaakt worden uit de hoeveelheid mensuren.
\chapter{Risico's} \label{cha:risicos}

Zoals bij elk project heeft dit project ook risico’s, deze risico’s zouden het project kunnen beïnvloeden waardoor het gedeeltelijk of geheel kan falen. De risico’s die zich bij dit project mogelijk kunnen voordoen zijn:

\section{Interne risico's} \label{sec:Interne risicos}

% Inserting a table

    \begin{tabular}{ | l | l | l | }
    	\hline
    	Nr. & Algemeen & Schaal(1 t/m 10) \\ \hline
    	1.	& Ziekteverzuim & 8 \\ \hline
    	2.  & Afwijken van de planning (tijdsnood) & 5 \\ \hline
    	3.  & Conflicten binnen de projectgroep & 7 \\ \hline
    	4.  & Miscommunicatie binnen de projectgroep & 6 \\ \hline
    	5.  & Product niet volgens de eisen van de opdrachtgever & 5 \\ \hline
  	\end{tabular}
\\
\\
    \begin{tabular}{ | l | l | l | }
    	\hline
    	Nr. & Kennis & Schaal(1 t/m 10) \\ \hline
    	1.	& Te weinig kennis over het project & 4 \\ \hline
    	2.  & Te weinig ervaring & 5 \\ \hline
  	\end{tabular}
\\
\\
    \begin{tabular}{ | l | l | l | }
    	\hline
    	Nr. & Technisch & Schaal(1 t/m 10) \\ \hline
    	1.	& Instabiele werkomgeving & 3 \\ \hline
    	2.  & Data verlies & 5 \\ \hline
    	3.  & Data niet bereikbaar & 7 \\ \hline
  	\end{tabular}

\section{Externe risico's} \label{sec:Externe risicos}

    \begin{tabular}{ | l | l | l | }
    	\hline
    	Nr. & Algemeen & Schaal(1 t/m 10) \\ \hline
    	1.	& Afwezigheid van de opdrachtgever & 4 \\ \hline
    	2.  & Miscommunicatie tussen de opdrachtgever en de projectgroep & 6 \\ \hline
  	\end{tabular}
\\
\\
Opmerking: Om zo min mogelijk door risico’s het project te laten beïnvloeden is er in de planning extra ruimte gecreëerd om eventuele achterstanden in te halen.


% The bibliography is printed with \bibliography{}. With the command \bibliographystyle{} a style is picked.
\bibliographystyle{plain}
\bibliography{refs/references}

% To close your document, add the \end{document} command. Everything after this command will not be processed.
\end{document}
